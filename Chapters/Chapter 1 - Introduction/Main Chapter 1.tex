\newpage
\section{Neuromorphic Engineering: History, objectives and challenges}

The field of Neuromorphic Engineering emerged in Caltech in the mid to late 1990's mainly through the work of Carver Mead, an eminent researcher who has enormously contributed to computing development and research. To understand what this field is all about, one should first look back at the birth of modern Neuroscience as well as the development of modern computing: both these fields witnessed major developments between the 1950s and 1970s w opened, as we all know, an ocean of opportunities for innovation. 

\subsection{Modern Neuroscience}

The field of Neuroscience is certainly the most misunderstood of all medical fields. The main reason is that its serious study only started being possible very late, as there was no way for researchers to observe anything of interest inside a dead brain, and let's not even talk about the possibility of observing things in alive beings. So much that some early views regarded the brain as "cranian stuffing". Egyptians believed that the heart was the seat of \textit{intelligence}, so they literally took out some parts of the brain (through nostrils) - yes, that was a long time ago, but that tells you something about how ignorant people were about the most complex human organ. 

Fast forward to the 20th century, where it was commonly accepted that the brain was where everything happened in terms of human senses and intelligence, and where Neuroscience started to be recognized as an academic discipline in its own right.

\subsubsection{Hodgkin and Huxley: The neural cell as a }
\subsubsection{Hubel and Wiesel}
\subsubsection{Model of the brain as a complex system with detailed individual component behaviour}

\subsection{Modern Computing}
\subsubsection{The transistor}
\subsubsection{Perspective shifts and realizations: Von Neuman, Mravin Minsky, Feynmann and Carver Mead}

\subsection{Neuromorphic Engineering}
\subsubsection{Carver Mead and the Caltech Graduate Course}
\subsubsection{First Breakthroughs: Misha Mahowald and Silicon Retina}
\subsubsection{From research breakthroughs to deliverables}

\subsection{The rise of Machine Learning}

\subsection{Modern challenges and objectives}
\subsubsection{Energy considerations}
\subsubsection{Efficiency considerations}
\subsection{Challenges}





\subsection{List of complementary readings}
\begin{itemize}
    \item Von Neumann Silliman Lecture: The Computer and the Brain. \footnote{\url{https://complexityexplorer.s3.amazonaws.com/supplemental_materials/5.6+Artificial+Life/The+Computer+and+The+Brain_text.pdf}}. This essay represents to me the birth of Computational Neuroscience. Von Neumann, saw and wrote (from his deathbed) the strinking analogies that one could make between the brain and the computer, and introduced the world to the idea of studying them both at the same time. 
    \item Analog VLSI and Neural Systems, Carver Mead \footnote{\url{https://www.amazon.com/Analog-VLSI-Neural-Systems-Carver/dp/0201059924}}. The first textbook on Neuromorphic Engineering, by the creator of the field. This textbook is a marvel (see Chapter 0) and explains so many critical concepts of the field with stunning simplicity. 
    \item Documentary on Misha Mahowald \footnote{\url{https://www.youtube.com/watch?v=lwT1jUvwRLc}}. Great overview of the development of the field and the kind of work that emerged from the "Physics of Computation" course at Caltech. 
    \item Giacomo's Porto conference \footnote{\url{https://youtube.com/watch?v=cwQ8edHQFOA}}. Great introduction to the motivation, breakthroughs and challenges of modern Neuromorphic Computing by Giacomo - our teacher and a pioneer in the field. 
    \item Shi-Chii's Conference \footnote{\url{https://www.youtube.com/watch?v=tZM49YjUVDk}}. Similarly to Giacomo's conference, great introduction to the modern challenges and applications of the field. 
    \item Event Based Vision: a survey. \footnote{https://arxiv.org/abs/1904.08405} Overview of one of the main applications of Neuromorphic Computing in a very comprehensive paper. 
    \item Carver Mead IEEE guest paper \footnote{\url{https://authors.library.caltech.edu/53090/1/00058356.pdf}}. This is an opinion paper from Carver Mead where he explains why he believes in the field having such a strong potential. 
    \item Hubel and Wiesel paper \footnote{\url{https://authors.library.caltech.edu/53090/1/00058356.pdf}} and documentary video \footnote{\url{https://www.youtube.com/watch?v=8VdFf3egwfg}}. Their experiments are revolutionary and it is worth knowing what they're about. 
\end{itemize}