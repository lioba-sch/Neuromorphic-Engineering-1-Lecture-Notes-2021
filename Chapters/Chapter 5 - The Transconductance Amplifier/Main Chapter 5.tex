\newpage
\section{The Transconductance Amplifier}

If a whole lecture is dedicated to this specific circuit, it is for a good reason. It is a building block of many different electrical circuits, within and beyond Neuromorphic Engineering. You may have heard of its cousin the Operational Amplifier (commonly called OpAmp), which is not that different. In this chapter, we'll first look at the architecture of this circuit, and then look at precise function with different case scenario. We'll finish by briefly looking at another circuit that is built from the transconductance amplifier: the Wide-output-range differential transconductance amplifier. 

Here are things you should be comfortable with before starting to read through this chapter: 
\begin{itemize}
    \item Architecture and behaviour of current mirror (specifically P-Type)
    \item Architecture and behaviour of differential-pair. 
    \item Architecture and behaviour or the diode connected transistor (specifically P-Type)
    \item Early Voltage and output conductance of transistors. 
\end{itemize}

NEED TO MENTION CONDUCTANCE OF TRANSISTORS.

\subfile{Architecture.tex}
\subfile{Function.tex}
\subfile{Wide Range.tex}
\subfile{Test Yourself.tex}
