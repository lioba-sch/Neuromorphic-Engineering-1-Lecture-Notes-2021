\newpage
\section{The Essential Physics Behind the Transistor}\label{chapter:transistor}
 As the main purpose of NE1 is to understand how to emulate brain function through elaborate transistor circuits, it is only necessary to grasp the intuition behind their intrinsic physical behaviour, and not the details of the physics, which would need a whole semester (at least) to cover properly. This is why we only focus on \textit{grossly} explaining the concepts in a digestible manner, with only the details that will be absolutely necessary to understand later concepts. This chapter is subsequently devoted to presenting all the critical building blocks of the transistor, which will be covered in the following chapter.
 
Before starting this chapter, the student should be familiar with the following concepts: 
\begin{itemize}
    \item Basic atomic structure (high school level)
    \item All Electrical Engineering fundamentals introduced in Chapter 0. 
\end{itemize}

\subfile{Silicon.tex}
\subfile{Diode.tex}
\subfile{MIS Capacitance.tex}
\subfile{Test Yourself.tex}


% This video presents an amazing water analogy to the idea of the transistor. Definitely something to use!
% https://www.youtube.com/watch?v=ELmcTg22HfY

