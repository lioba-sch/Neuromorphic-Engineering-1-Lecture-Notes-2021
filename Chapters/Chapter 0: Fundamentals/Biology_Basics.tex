\section{Basics of Biology}

\subsubsection{The biological cell, its resistance and capacitance}

\paragraph{Biological cells and capacitance}\footnote{https://www.scientifica.uk.com/learning-zone/understanding-the-cell-as-an-electrical-circuit}

The cell membrane consists of a double lipid layer that separates ions in the extracellular space from ions and charged proteins in the cytoplasm. While pure lipid membranes are excellent electrical insulators, real cell membranes consist of a dense mosaic of proteins and lipids. Many of these proteins span the membrane and act as channels that allow charge to pass. These proteins reduce the otherwise high resistance of the membrane, which has significant consequences for electrophysiology. Assume we want to apply a voltage across the cell membrane by injecting current with an electrode. The current required to maintain this voltage is determined by the membrane resistance, according to Ohm's Law: Voltage = Resistance * Current (or V = R * I). We can see that the higher the membrane resistance, the lower the current required to maintain a given membrane voltage.