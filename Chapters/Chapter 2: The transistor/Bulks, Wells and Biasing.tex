\subsection{Bulks, wells and biasing the MOSFET Bulks}
\newline \newline
\textbf{What's up with bulks and wells:} \\
Good question. A lot is actually going on, but let's try to keep it short and simple. Remember from chapter 0 (which I hope you reviewed!) that voltage is all about potential \emph{difference}, so you need "point of references" that are somewhat common to everywhere in your circuit. It so happens that in a circuit, all transistors do not necessarily share the same bulk, which is typically the natural reference potential of MOSFETS. We therefore need a common reference potential that multiple transistors can access to: this is called the common bulk potential. With nFETs, it's called $V_{ss}$, and with pFETs, it's called $V_{dd}$. Now if you have large currents getting from MOSFETs going straight into the bulks, that may create some problems (I'm guessing altering the potential). To avoid that, source and drain diodes to the bulks are reverse-biased. Remember from Chapter 0 that a diode only lets current flowing in one direction, until breaking point! So no current flows into the bulk, only out of the bulk. Now we have a point where you can safely say that $V_{ss}$ is the lowest potential in the circuit and $V_{dd}$ is the highest - they're called \emph{power rails}. All voltages are references to $V_{ss}$ so to be positive. Connections to $V_{ss}$ are marked as connection to ground, and connections to $V_{dd}$ with a slanting line. 
\newline \newline
\textbf{Biasing MOSFET bulks:} \\
First, let's look at how an nFET should be biased. The drain voltage, $V_d$, and the source voltage, $V_s$, of an nFET (see Fig. 3.2(a)) should be greater than or equal to the bulk voltage, $V_b$, so that the PN junctions between the highly doped $n^+$ regions and the substrate will be reverse biased. That is, $V_{sb} = V_s - V_b \geq  0$ and$V_{db} = V_d - V_b \geq  0$. These bias conditions guarantee that there will only be a small reverse leakage current at these junctions and that most of the transistor’s current will flow in the channel. In an nFET, the $n^+$
region biased at the higher voltage is called the drain, and the other $n^+$ region is called the source. Because electrons are negatively charged, the direction of positive current flow, I , is from drain to source even though the carriers (electrons) flow from source to drain. The currents measured at the source and at the drain are approximately the same, that is, there is very little loss of carriers along the channel. \ 
For a pFET, the $p^+$ regions should be biased negative relative to its bulk, that is, $V_{sb} = V_s - V_b \leq  0$ and $V_{db} = V_d - V_b \leq  0$ so that the PN junctions are again reverse-biased. The n-type bulk (or n-well) of the pFET should be
biased higher than the p-substrate. For a pFET, the $p^+$ region which is biased at the higher voltage is called the source and the other $p^+$ region is called the
drain. Because holes are positively charged, positive channel current, I , flows from the source to the drain. The bulk of the pFET is usually connected to the highest voltage ($V_{dd}$) supplied to the chip while the bulk of the nFET is tied to the lowest voltage ($V_{ss}$). In a p-substrate, where the pFET rests in an n-well, the substrate is connected to $V_{ss}$,
and the well to $V_{dd}$.

We can Bias the MOSFETs to different reference voltages, which create different dynamics. Some circuits that will be studied in this course, such as the pFET source-follower, has a bulk connected to its output voltage, which for some reason I do not yet understand, removes the $\kappa$ component of its output voltage. 