\subsection{Exam Tips}
\subsubsection{General comments on the exam format and expected level of details}
\subsubsection{Questions that they told us were popular}
\subsubsection{Questions that they we think will be asked}
\begin{itemize}
    \item How do you measure $\kappa$ if you don't have access to the physics of the devices? \\
    You could use a NFET source follower, which transfer function includes Kappa, two voltages which one can control (from current source and the input transistor voltage) and one output voltage that you can measure. 
    \item You could be given a basic circuit with example voltages and ask to derive the voltage/current of a given element, or simply say if everything makes sense or if we're violating some assumption like saturation.
\end{itemize}
\subsubsection{How to practice your circuits}
\begin{itemize}
    \item Reason your way through circuits, with voltages/current starting point and everything that follows from this. This was done on the photodiode chapter. Everytime you can take assumptions such as $\kappa_n \mathrm{and} \kappa_p = 1$, $I_{0_n} = I_{0_p}$, all transistors have the same $W_L$, neglect the Early Effect (except if important such as in small signal difference WTA for example).  
\end{itemize}

